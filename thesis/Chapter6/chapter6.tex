%*******************************************************************************
%****************************** Second Chapter *********************************
%*******************************************************************************

\chapter{Conclusions and Future work}

\ifpdf
    \graphicspath{{Chapter6/Figs/Raster/}{Chapter6/Figs/PDF/}{Chapter6/Figs/}}
\else
    \graphicspath{{Chapter6/Figs/Vector/}{Chapter6/Figs/}}
\fi

Over the last few decades Machine Learning has become one of the central component of information technology, though mostly hidden part of our life. The increasing availability of very high dimensional data, with diverse characteristics and growing complexity, there is good reason to believe that smart data analysis has became a necessary ingredient for technological progress and achieve the wisdom. Machine learning is a joint field of artificial intelligence and modern statistics, mostly focused on the design and development of models, algorithms and techniques to extract information automatically from data. Data modelling with Gaussian process is the state-of-the-art technique in the wider community, and in practice turned to multidisciplinary. Our main focus of this thesis was to achieve few goals by building Gaussian process models on transcriptome data and analyse their behaviour. Here, the final chapter of this thesis aim to summarise the key ideas and main contributions of the previous chapters and consider possible direction of future work.


\section{Summary of the specific Contributions}
\textbf{Chapter 2:} We have reimplemented the tool \emph{chipDyno} using \emph{R} programming language with the aim to make it public through a open source platform via GitHub. We constructed our connectivity information between genes and transcription factors from the evidence of gene to gene interaction to model the TFAs. Earlier the dynamics of TFAs were obtained for a unicellular microorganism (yeast), our tool modelled transcription factor activities for a multicellular eukaryote (\textit{C.elegans}). The probabilistic dynamical model for quantitative inference of TFA, that we described in this chapter used as the basis for the  Chapter \ref{ch:GaussianProcessRegression} and Chapter \ref{ch:GP_Model_of_TFAs}.

\textbf{Chapter 3:} In this technical background chapter we briefly described Gaussian processes, the regression problem and regression with Gaussian processes. The choice of the covariance function is a central step in modelling with a Gaussian process. We justified the rationale behind choosing the Ornstein-Uhlenbeck kernel to model the transcription factor activity using Gaussian processes. This analogy leads us to develop a special covariance functions suitable for transcription factor activity analysis. 

\textbf{Chapter 4:} The linear Gaussian model is equivalent to a Gaussian process with a particular covariance function. We therefore built a model directly from the Gaussian process perspective. Here we designed a covariance function for reconstructing transcription factor activities given gene expression profiles and a connectivity information between genes and transcription factors. The joint process across all transcription factor activities and across all time points might have some correlation, here we incorporated intrinsic model of coregionalization for the the joint process. We also introduced a computational trick, based on  judicious application of singular value decomposition, to enable us to efficiently fit the Gaussian process in a reduced \lq TF activity\rq space. 

\textbf{Chapter 5:} We have performed genome-wide analysis to cluster genes systematically and analyse the rationale behind the variation in the speed of propagation for ALS. Our particular innovation was to include the condition and genetic background of the organisms within the underlying functional component of our clusters. This ensured that the underlying expressions behaved similarly were more likely to be clustered together. We used a widely acceptable gene ontology and functional annotation tool to validate our clusters and their characteristics obtained from our model. Gene expression time series characteristics curve and enrichment scores analyse helped us to narrow down our search and lead toward finding the lists of genes or clusters which could be involved in the speed of disease propagation. Our pathway analysis found evidence of genes which are known to be involved in the disease process. The special covariance function we have developed for clustering considering models condition, genetic background, replicates and disease states with coregionalization could be useful to investigate other biological activity where clustering is required. 

\section{Future Work}
Here we are going to set some possible directions of future work

\textbf{Bridge between TFA and clustering:} In Chapter \ref{ch:Probabilistic_TFA}, we developed a model to analyse a latent variable (transcription factor) and determined their dynamic behaviour using the gene expression time series data. In Chapter  \ref{ch:Clustering_Gene_Expression_Data}, we developed another model to cluster  gene expressions considering genetic background, model conditions and replicates. We aim to build a model \lq as a whole \rq, which will model the dynamics of latent factor (transcription factor) considering various genetic backgrounds, conditions and replicates and hence cluster the gene expressions based on both of the latent factors and shared information.

\textbf{Validation of clustered genes:} Differential Multi Information (DMI) (\cite{Gambardella:2015}) value indicates the level of differential co-expression of the gene set among the two classes. Each gene set is associated to a DMI value, computed as the absolute difference of the R{\'e}nyi mutual information (RMI) (\cite{Renyi:1960}) among diseases and among controls. An high DMI value means that the same genes are co-expressed (and thus co-regulated) in a different manner between the conditions. The gene set is for this reason addressed as ``differentially co-expressed'', and considered relevant for the analysed disease. On the contrary, if a gene set has a low DMI value, it means that the co-expression of the genes in the two biological conditions is almost the same, thus the gene set is not relevant for the analysed disease since different biological conditions do not seem to affect the co-expression of that particular gene set. One of our future plan is to validate the cluster of genes by building a model with using DMI and RMI, and calculate the associativity.


\textbf{Big Data:} 
In Chapter \ref{ch:GaussianProcessRegression} we addressed data with higher number of features might be an issue while modelling with Gaussian process. On the other side, due to advancement of data acquisition techniques, every day the amount of data increasing tremendously. These data are well know by a fancy term \emph{Big Data}. Knowledge extraction and interpretation of \emph{Big Data} is a new challenge, which also triggered the demand of special algorithms or models. The generic inference and learning algorithms in Gaussian processes where we need to inverse the matrix has $\mathcal{O}\left(N^3\right)$ runtime complexity and $\mathcal{O}\left(N^2\right)$ memory complexity. However, an increasing number of machine learning research (\cite{Hensman:2013a, Dai:2014}) has focused to overcome these problems even with $N>10^6$ data size. Our clustering algorithm (we described in Chapter \ref{ch:Clustering_Gene_Expression_Data}) targets multiple model conditions where data size may grow geometrically. We aim to extend our clustering model presented in this thesis (Chapter \ref{ch:Clustering_Gene_Expression_Data}), which will able to handle the \emph{Big Data}.  


\textbf{Deep learning:} ?

\textbf{EP:} ?

\textbf{Application:} ?


