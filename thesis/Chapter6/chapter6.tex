%*******************************************************************************
%****************************** Second Chapter *********************************
%*******************************************************************************

\chapter{Conclusion and Future work}

\ifpdf
    \graphicspath{{Chapter6/Figs/Raster/}{Chapter6/Figs/PDF/}{Chapter6/Figs/}}
\else
    \graphicspath{{Chapter6/Figs/Vector/}{Chapter6/Figs/}}
\fi


We have developed a tool based on programming language \emph{R} named \emph{chipDyno} using the model of
\cite{Sanguinetti:2006} which integrate the connectivity information between genes and transcription
factors, and micro array data. The probabilistic nature of the model can determine the significant
regulations in a given experimental condition.

Earlier the model was developed for a unicellular microorganism (yeast) but we have successfully
manage to determine the gene specific transcription factor activity for \textit{C. elegans}, a
multicellular eukaryote. We were also successful to filter out the quiet genes from the
differentially expressed genes.

To elucidate pathways and processes relevant to human biology 
and disease \textit{C. elegans} is been using as a vital model. 
Different orthology-prediction methods (\cite{Daniel:2011}) are using 
to compile a list of \textit{C. elegans} orthologs of human genes. Already 
a  list of 7,663 unique protein-coding genes were resulted in that list and this
represents ~38\% of the 20,250 protein-coding genes predicted in \textit{C. elegans}. 
When human genes introduced into \textit{C.Elegans} human genes replaced their homologous. 
On the contrary, many \textit{C. elegans}
genes can function with great deal of similarity to human like mammalian genes. So, 
the biological insight acquire from \textit{C. elegans} may be directly applicable to more 
complex organism like human.

Lots of computational approaches on gene expression data for time series analysis are not
well suited where time points are irregularly spaced. Even in commonly used state-space
model time points must occur at regular intervals. On the other side gene expression
experiments with regular samples may not be cost effective or optimal from the perspective of
statistics. It is expected that models with irregular time points might be more informative if 
the time points are selected considering some temporal features. Gaussian process is not 
restricted to equally spaced time series data. Already Gaussian process regression have been 
successfully applied to overcome this issue and analyse time series data (\cite{Kalaitzis:2011}).
So our expected model will overcome the restriction of temporal sampling of equally spaced
time intervals. 
%----------------------------------------------------------------------------------------

\section{Future Work}
\cite{Sanguinetti:2006} model to infer the transcription factor 
activity is a linear- Gaussian state-space model. We believe that this linear Gaussian
model is equivalent to Gaussian process with a specific covariance function.
We have developed a model directly from Gaussian process to achieve the same goal.
We are quite close to develop a valid covariance function for reconstructing transcription
factor activities given gene expression profile and binding information between genes and
transcription factors. Here we will introduce a computational trick using
singular value decomposition and intrinsic coregionalization model. We believe
this method will enable us to efficiently fit the Gaussian process in a
reduced transcription factor activity space.

Amyotrophic lateral sclerosis (ALS), also known as ``Lou Gehrig's Disease'' or motor neurone disease 
(MND), is an irreversible progressive neurodegenerative adult onset that affects motor neurons 
in the brain and the spinal cord. Muscle denervation spreads over neuromuscular system and 
leads toward death by failure of the respiratory system with in few years of system 
onset (\cite{Peviani:2010}). 
This lethal invariable disorder has median survival of less than 5 years, only 20\% 
of the affected people can survive more than 5 years and 10\% of the patients 
can survive more than 10 years. Mutation in the Cu/Zn superoxide dismutase (SOD1) gene 
is responsible for around 20\% of the familial motor neurone disease \cite{Nardo:2013}. 
Transgenic mice can express human SOD1 mutation and nicely replicates different 
histopathological and clinical features of motor neurone disease. Mimic of these murine models 
of different clinical phenotypes observed from human MND patients are widely using by the 
researchers to determine the disease progression. But we didn't found any evidence of gene expression 
analysis that attempt to analyse motor neuron disease considering the genetic background on 
different phenotype of this murine disease. So gene expression analysis for different murine models 
could reveal interesting information. 
\cite{Nardo:2013} used two mouse models to analyse fast and slow disease progression of ALS.
We will use our Gaussian process based model to infer the 
transcription factor activity on gene expression data obtained from different murine models and 
try to find out some fascinating insights.

Clustering of gene expression time series is another major interest of the research to get the view of
groups of co-regulated or associated genes. It is assumed that gene involved in the same biological 
process will be expressed with a similarity sharing underlying time series. \cite{Cossins:2007}
did some additional cluster analysis (not published yet!) based on some phenotype properties. Again 
it is very common to have multiple biological replicates of the gene expression time series data. 
Just taking average of the replicates surely lead toward discarding insight. Recently \cite{Hensman:2013} used
a hierarchy of Gaussian process to model a gene specific and replicate specific temporal covariance.
They also used this model for clustering application. Using this Gaussian process based hierarchical
clustering analysis of \cite{Hensman:2013} we will try to find some robust clusters for the gene
expression data of \textit{C. elegans}. Once if we can do so, it will easily lead us to find out the active
transcription factors related with these clusters and their subsequent dynamic behaviour as well.