% ************************** Thesis Abstract *****************************
% Use `abstract' as an option in the document class to print only the titlepage and the abstract.
\begin{abstract}
In molecular biology and genetics, a transcription factor is a protein that binds to specific DNA
sequences and controls the flow of genetic information from DNA to mRNA. To develop models of 
cellular processes, quantitative estimation of the regulatory relationship between 
transcription factors and genes is a basic requirement. But quantitative estimation is complex 
due to some reasons. Many of the transcription factors' activity and their own transcription 
level are post transcriptionally modified; very often the levels of the transcription factors' 
expressions are low and also contain noise. So, from the expression levels of their target 
genes it is useful to infer the activity of the transcription factors. Here we develop a 
Gaussian process based regression to infer the exact TFAs from a 
combination of mRNA expression level and DNA protein binding measurement.


  Clustering of gene expression time series gives insight into which genes may be coregulated, allowing us to discern the activity of pathways in a given microarray experiment. Of particular interest is how a given group of genes varies with different conditions or genetic background.

  In this paper we develop a new clustering method that allows each cluster to be parameterised according to whether the behaviour of the genes across conditions is correlated or anti-correlated. By specifying correlation between such genes we gain more information within the cluster about how the genes interrelate.

Amyotrophic lateral sclerosis
(ALS) is an irreversible neurodegenerative disorder that kills the
motor neurons and results in death within 2 to 3 years from the
symptom onset.  Speed of progression for different patients are
heterogeneous with significant variability.  It is already
reported that $SOD1^{G93A}$ transgenic mice from different backgrounds
($129Sv$ and $C57$) showed consistent phenotypic differences for
disease progression.  Here we used a hierarchy of Gaussian processes
 to model condition-specific and gene-specific
temporal covariances. This study demonstrated about finding some
significant gene expression profiles and clusters of associated or
co-regulated gene expressions together from four groups of data
($SOD1^{G93A}$ and $Ntg$ from $129Sv$ and $C57$ backgrounds).  Further
gene enrichment score analysis and ontology pathway analysis of some
specified clusters for a particular group may lead toward identifying
features underlying the differential speed of disease progression. Our
study shows the effectiveness of sharing information between
replicates and different model conditions when modelling gene
expression time series.


\end{abstract}
