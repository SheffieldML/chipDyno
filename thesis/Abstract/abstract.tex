% ************************** Thesis Abstract *****************************
% Use `abstract' as an option in the document class to print only the titlepage and the abstract.
\begin{abstract}
In the field of machine learning, Gaussian process models are widely used families of stochastic processes for modelling data observed over time, or space and time or space. Gaussian processes models are nonparametric, meaning that models are developed on an infinite-dimensional parameter space. The parameter space is typically learn as the set of all possible solutions for a given learning problem. Gaussian process distributions are distribution over function. The covariances function determines the properties of sample function drawn from the distribution. Once the decision to model with a Gaussian process has been made the choice of the covariance function is a central step in modelling.

In molecular biology and genetics, a transcription factor is a protein that binds to specific DNA sequences and controls the flow of genetic information from DNA to mRNA. To develop models of cellular processes, quantitative estimation of the regulatory relationship between transcription factors and genes is a basic requirement. Quantitative estimation is complex due to various reasons. Many of the transcription factors' activity and their own transcription level are post transcriptionally modified; very often the levels of the transcription factors' expressions are low and noisy. So, from the expression levels of their target genes it is useful to infer the activity of the transcription factors. Here we developed a Gaussian process based nonparametric regression model to infer the exact transcription factor activities from a combination of mRNA expression level and DNA protein binding measurement.

Clustering of gene expression time series gives insight into which genes may be coregulated, allowing us to discern the activity of pathways in a given microarray experiment. Of particular interest is how a given group of genes varies with different conditions or genetic backgrounds. In this dissertation we developed a new clustering method that allows each cluster to be parametrised according to the behaviour of the genes across conditions whether they are correlated or anti-correlated. By specifying correlation between such genes we gain more information within the cluster about how the genes interrelate. Our study shows the effectiveness of sharing information between replicates and different model conditions while modelling gene expression time series.

\end{abstract}
